% basics
\usepackage[utf8]{inputenc}
\usepackage[T1]{fontenc}
\usepackage{textcomp}
% \usepackage[dutch]{babel}
\usepackage{url}
% \usepackage{hyperref}
% \hypersetup{
%     colorlinks,
%     linkcolor={black},
%     citecolor={black},
%     urlcolor={blue!80!black}
% }
\usepackage{graphicx}
\usepackage{float}
\usepackage{booktabs}
\usepackage{enumitem}
% \usepackage{parskip}
\usepackage{emptypage}
\usepackage{subcaption}
\usepackage{multicol}
\usepackage[usenames,dvipsnames]{xcolor}

% \usepackage[none]{hyphenat}

% \usepackage{cmbright}


\usepackage{amsmath, amsfonts, mathtools, amsthm, amssymb}
\usepackage{physics}
\usepackage{gensymb}
\usepackage{tabularray}
\usepackage{mathrsfs}
\usepackage{cancel}
\usepackage{bm}
\newcommand\N{\ensuremath{\mathbb{N}}}
\newcommand\R{\ensuremath{\mathbb{R}}}
\newcommand\Z{\ensuremath{\mathbb{Z}}}
\renewcommand\O{\ensuremath{\emptyset}}
\newcommand\Q{\ensuremath{\mathbb{Q}}}
\newcommand\C{\ensuremath{\mathbb{C}}}
\DeclareMathOperator{\sgn}{sgn}
\usepackage{systeme}
\let\svlim\lim\def\lim{\svlim\limits}
\let\implies\Rightarrow
\let\impliedby\Leftarrow
\let\iff\Leftrightarrow
	\let\epsilon\varepsilon
	\usepackage{stmaryrd} % for \lightning
	\newcommand\contra{\scalebox{1.1}{$\lightning$}}
	% \let\phi\varphi


	% Don't indent paragraphs, leave some space between them

	\usepackage{parskip}


	% correct
	\definecolor{correct}{HTML}{009900}
	\newcommand\correct[2]{\ensuremath{\:}{\color{red}{#1}}\ensuremath{\to }{\color{correct}{#2}}\ensuremath{\:}}
	\newcommand\green[1]{{\color{correct}{#1}}}

	% horizontal rule
	\newcommand\hr{
		\noindent\rule[0.5ex]{\linewidth}{0.5pt}
	}

	% derivatives

	\newcommand{\dx}{\,\mathrm{d}x}
	\newcommand{\dy}{\,\mathrm{d}y}
	\newcommand{\dz}{\,\mathrm{d}z}

	% hide parts
	\newcommand\hide[1]{}

	% si unitx
	\usepackage{siunitx}
	\sisetup{locale = FR}
	% \renewcommand\vec[1]{\mathbf{#1}}
	\newcommand\mat[1]{\mathbf{#1}}


	% tikz
	\usepackage{tikz}
	\usepackage{tikz-cd}
	\usetikzlibrary{intersections, angles, quotes, calc, positioning}
	\usetikzlibrary{arrows.meta}
	\usetikzlibrary{babel}
	\usepackage{pgfplots}
	\pgfplotsset{
		compat=newest,
		table/search path={build/},
	}

	\tikzset{
		force/.style={thick, {Circle[length=2pt]}-stealth, shorten <=-1pt}
	}

	\newcounter{examplecounter}[section]
	\newcounter{problemcounter}[section]

	% Syntax: \colorboxed[<color model>]{<color specification>}{<math formula>}
	\newcommand*{\colorboxed}{}
	\def\colorboxed#1#{%
	\colorboxedAux{#1}%
	}
	\newcommand*{\colorboxedAux}[3]{%
		% #1: optional argument for color model
		% #2: color specification
		% #3: formula
		\begingroup
		\colorlet{cb@saved}{.}%
		\color#1{#2}%
		\boxed{%
			\color{cb@saved}%
			#3%
		}%
		\endgroup
	}

	\makeatletter
	\newcommand*\Acolorboxed[2][red]{%
		\let\bgroup{\romannumeral-`}%
		\@Acolorboxed{#1}#2&&\ENDDNE
	}
	\def\@Acolorboxed#1#2&#3&#4\ENDDNE{%
		\ifnum0=`{}\fi
		\setbox\z@\hbox{$\displaystyle#2{}\m@th$\kern\fboxsep \kern\fboxrule}%
		\edef\@tempa{\kern\wd\z@ & \kern-\the\wd\z@ \fboxsep\the\fboxsep \fboxrule\the\fboxrule}%
		\@tempa
		\fcolorbox{#1}{white}{\m@th$\displaystyle#2#3$}%
	}
	\makeatother

	% theorems
	\makeatother
	\usepackage{thmtools}
	\usepackage[framemethod=TikZ]{mdframed}
	\mdfsetup{skipabove=1em,skipbelow=0em}

	% themes
	\newcommand\definitioncolor{Aquamarine}
	\newcommand\examplecolor{Melon}
	\newcommand\problemcolor{RubineRed}
	\newcommand\propositioncolor{Plum}
	\newcommand\theoremcolor{SeaGreen}
	\newcommand\lemmacolor{Periwinkle}
	\newcommand\corollarycolor{Dandelion}
	\newcommand\solutioncolor{Melon}
	\newcommand\proofcolor{Orchid}

	\theoremstyle{definition}

\declaretheoremstyle[
	headfont=\bfseries\sffamily\color{\definitioncolor!70!black}, bodyfont=\normalfont,
	mdframed={
			linewidth=1pt,
			leftline=true, rightline=false, topline=false, bottomline=false,
			linecolor=\definitioncolor!30, backgroundcolor=\definitioncolor!1,
			splitbottomskip=3em,
			splittopskip=0em,
			afterbreak=\ifmdffirstframe{\begin{minipage}{\linewidth}\vspace{-1.5em}\color{\definitioncolor!70!black}\centering $\bm{\cdots}$\end{minipage}}\fi,
			beforelastframe={\begin{mdframed}[innerbottommargin=2em]\color{\definitioncolor!70!black}\centering $\bm{\cdots}$\end{mdframed}\vspace{-0.1em}},
		}
]{thmdefinitionbox}

\declaretheoremstyle[
	headfont=\bfseries\sffamily\color{\examplecolor!70!black}, bodyfont=\normalfont,
	mdframed={
			linewidth=1pt,
			leftline=true, rightline=false, topline=false, bottomline=false,
			linecolor=\examplecolor!30, backgroundcolor=\examplecolor!3,
			splitbottomskip=3em,
			splittopskip=0em,
			afterbreak=\ifmdffirstframe{\begin{minipage}{\linewidth}\vspace{-1.5em}\color{\examplecolor!70!black}\centering $\bm{\cdots}$\end{minipage}}\fi,
			beforelastframe={\begin{mdframed}[innerbottommargin=2em]\color{\examplecolor!70!black}\centering $\bm{\cdots}$\end{mdframed}\vspace{-0.1em}},
		}
]{thmexamplebox}

\declaretheoremstyle[
	headfont=\bfseries\sffamily\color{\problemcolor!70!black}, bodyfont=\normalfont,
	mdframed={
			linewidth=1pt,
			leftline=true, rightline=false, topline=false, bottomline=false,
			linecolor=\problemcolor!30, backgroundcolor=\problemcolor!3,
			splitbottomskip=3em,
			splittopskip=0em,
			afterbreak=\ifmdffirstframe{\begin{minipage}{\linewidth}\vspace{-1.5em}\color{\problemcolor!70!black}\centering $\bm{\cdots}$\end{minipage}}\fi,
			beforelastframe={\begin{mdframed}[innerbottommargin=2em]\color{\problemcolor!70!black}\centering $\bm{\cdots}$\end{mdframed}\vspace{-0.1em}},
		}
]{thmproblembox}

\declaretheoremstyle[
	headfont=\bfseries\sffamily\color{\propositioncolor!70!black}, bodyfont=\normalfont,
	mdframed={
			linewidth=1pt,
			leftline=true, rightline=false, topline=false, bottomline=false,
			linecolor=\propositioncolor!30, backgroundcolor=\propositioncolor!3,
			splitbottomskip=3em,
			splittopskip=0em,
			afterbreak=\ifmdffirstframe{\begin{minipage}{\linewidth}\vspace{-1.5em}\color{\propositioncolor!70!black}\centering $\bm{\cdots}$\end{minipage}}\fi,
			beforelastframe={\begin{mdframed}[innerbottommargin=2em]\color{\propositioncolor!70!black}\centering $\bm{\cdots}$\end{mdframed}\vspace{-0.1em}},
		}
]{thmpropbox}

\declaretheoremstyle[
	headfont=\bfseries\sffamily\color{\proofcolor!70!black}, bodyfont=\normalfont,
	mdframed={
			linewidth=1pt,
			leftline=true, rightline=false, topline=false, bottomline=false,
			linecolor=\proofcolor!30, backgroundcolor=\proofcolor!1,
			splitbottomskip=3em,
			splittopskip=0em,
			afterbreak=\ifmdffirstframe{\begin{minipage}{\linewidth}\vspace{-1.5em}\color{\proofcolor!70!black}\centering $\bm{\cdots}$\end{minipage}}\fi,
			beforelastframe={\begin{mdframed}[innerbottommargin=2em]\color{\proofcolor!70!black}\centering $\bm{\cdots}$\end{mdframed}\vspace{-0.1em}},
		}
]{thmproofbox}

\declaretheoremstyle[
	headfont=\bfseries\sffamily\color{\theoremcolor!70!black}, bodyfont=\normalfont,
	mdframed={
			linewidth=1pt,
			leftline=true, rightline=false, topline=false, bottomline=false,
			linecolor=\theoremcolor!30, backgroundcolor=\theoremcolor!3,
			splitbottomskip=3em,
			splittopskip=0em,
			afterbreak=\ifmdffirstframe{\begin{minipage}{\linewidth}\vspace{-1.5em}\color{\theoremcolor!70!black}\centering $\bm{\cdots}$\end{minipage}}\fi,
			beforelastframe={\begin{mdframed}[innerbottommargin=2em]\color{\theoremcolor!70!black}\centering $\bm{\cdots}$\end{mdframed}\vspace{-0.1em}},
		}
]{thmtheorembox}

\declaretheoremstyle[
	headfont=\bfseries\sffamily\color{\lemmacolor!70!black}, bodyfont=\normalfont,
	mdframed={
			linewidth=1pt,
			leftline=true, rightline=false, topline=false, bottomline=false,
			linecolor=\lemmacolor!30, backgroundcolor=\lemmacolor!3,
			splitbottomskip=3em,
			splittopskip=0em,
			afterbreak=\ifmdffirstframe{\begin{minipage}{\linewidth}\vspace{-1.5em}\color{\lemmacolor!70!black}\centering $\bm{\cdots}$\end{minipage}}\fi,
			beforelastframe={\begin{mdframed}[innerbottommargin=2em]\color{\lemmacolor!70!black}\centering $\bm{\cdots}$\end{mdframed}\vspace{-0.1em}},
		}
]{thmlemmabox}

\declaretheoremstyle[
	headfont=\bfseries\sffamily\color{\corollarycolor!70!black}, bodyfont=\normalfont,
	mdframed={
			linewidth=1pt,
			leftline=true, rightline=false, topline=false, bottomline=false,
			linecolor=\corollarycolor!30, backgroundcolor=\corollarycolor!3,
			splitbottomskip=3em,
			splittopskip=0em,
			afterbreak=\ifmdffirstframe{\begin{minipage}{\linewidth}\vspace{-1.5em}\color{\corollarycolor!70!black}\centering $\bm{\cdots}$\end{minipage}}\fi,
			beforelastframe={\begin{mdframed}[innerbottommargin=2em]\color{\corollarycolor!70!black}\centering $\bm{\cdots}$\end{mdframed}\vspace{-0.1em}},
		}
]{thmcorolbox}

\usepackage[absolute,overlay]{textpos}

\declaretheoremstyle[
	headfont=\bfseries\sffamily\color{\solutioncolor!70!black}, bodyfont=\normalfont,
	numbered=no,
	mdframed={
			linewidth=1pt,
			leftline=true, rightline=false, topline=false, bottomline=false,
			linecolor=\solutioncolor!30, backgroundcolor=\solutioncolor!1,
			splitbottomskip=3em,
			splittopskip=0em,
			afterbreak=\ifmdffirstframe{\begin{minipage}{\linewidth}\vspace{-1.5em}\color{\solutioncolor!70!black}\centering $\bm{\cdots}$\end{minipage}}\fi,
			beforelastframe={\begin{mdframed}[innerbottommargin=2em]\color{\solutioncolor!70!black}\centering $\bm{\cdots}$\end{mdframed}\vspace{-0.1em}},
		},
	qed=\qedsymbol
]{thmsolutionbox}

\declaretheoremstyle[
	headfont=\bfseries\sffamily\color{NavyBlue!70!black}, bodyfont=\normalfont,
	mdframed={
			linewidth=1pt,
			leftline=true, rightline=false, topline=false, bottomline=false,
			linecolor=NavyBlue!30, backgroundcolor=NavyBlue!3,
			splitbottomskip=3em,
			splittopskip=0em,
			afterbreak=\ifmdffirstframe{\begin{minipage}{\linewidth}\vspace{-1.5em}\color{NavyBlue!70!black}\centering $\bm{\cdots}$\end{minipage}}\fi,
			beforelastframe={\begin{mdframed}[innerbottommargin=2em]\color{NavyBlue!70!black}\centering $\bm{\cdots}$\end{mdframed}\vspace{-0.1em}},
		}
]{thmbluebox}

\declaretheoremstyle[
	headfont=\bfseries\sffamily\color{NavyBlue!70!black}, bodyfont=\normalfont,
	mdframed={
			linewidth=1pt,
			leftline=false, rightline=false, topline=true, bottomline=true,
			linecolor=NavyBlue!7,backgroundcolor=NavyBlue!2,
			splitbottomskip=3em,
			splittopskip=0em,
			afterbreak=\ifmdffirstframe{\begin{minipage}{\linewidth}\vspace{-1.5em}\color{NavyBlue!70!black}\centering $\bm{\cdots}$\end{minipage}}\fi,
			beforelastframe={\begin{mdframed}[innerbottommargin=2em]\color{NavyBlue!70!black}\centering $\bm{\cdots}$\end{mdframed}\vspace{-0.1em}},
		}
]{thmblueline}

% \declaretheoremstyle[headfont=\bfseries\sffamily, bodyfont=\normalfont, mdframed={ nobreak } ]{thmgreenbox}
% \declaretheoremstyle[headfont=\bfseries\sffamily, bodyfont=\normalfont, mdframed={ nobreak } ]{thmredbox}
% \declaretheoremstyle[headfont=\bfseries\sffamily, bodyfont=\normalfont]{thmbluebox}
% \declaretheoremstyle[headfont=\bfseries\sffamily, bodyfont=\normalfont]{thmblueline}
% \declaretheoremstyle[headfont=\bfseries\sffamily, bodyfont=\normalfont, numbered=no, mdframed={ rightline=false, topline=false, bottomline=false, }, qed=\qedsymbol ]{thmproofbox}
% \declaretheoremstyle[headfont=\bfseries\sffamily, bodyfont=\normalfont, numbered=no, mdframed={ nobreak, rightline=false, topline=false, bottomline=false } ]{thmexplanationbox}

\declaretheorem[style=thmdefinitionbox, name=Definition]{definition}
\declaretheorem[style=thmexamplebox, name=Example, sibling=examplecounter]{eg}
\declaretheorem[style=thmproblembox, name=Problem, sibling=problemcounter]{prb}
\declaretheorem[style=thmpropbox, name=Proposition]{prop}
\declaretheorem[style=thmtheorembox, name=Theorem]{theorem}
\declaretheorem[style=thmlemmabox, name=Lemma]{lemma}
\declaretheorem[style=thmcorolbox, numbered=no, name=Corollary]{corollary}

\declaretheorem[style=thmproofbox, name=$\bm{\hookrightarrow}$ Proof]{replacementproof}
\renewenvironment{proof}[1][\proofname]{\vspace{-10pt}\begin{replacementproof}}{\end{replacementproof}}

\declaretheorem[style=thmsolutionbox, name=$\bm{\hookrightarrow}$ Solution]{tmpsolution}
\newenvironment{solution}[1][]{\vspace{-10pt}\begin{tmpsolution}~}{\end{tmpsolution}}

\declaretheorem[style=thmblueline, numbered=no, name=Remark]{remark}
\declaretheorem[style=thmblueline, numbered=no, name=Note]{note}

\newtheorem*{uovt}{UOVT}
\newtheorem*{notation}{Notation}
\newtheorem*{previouslyseen}{As previously seen}
\newtheorem*{problem}{Problem}
\newtheorem*{observe}{Observe}
\newtheorem*{property}{Property}
\newtheorem*{intuition}{Intuition}

\usepackage{etoolbox}
\AtEndEnvironment{vb}{\null\hfill$\diamond$}%
\AtEndEnvironment{intermezzo}{\null\hfill$\diamond$}%
% \AtEndEnvironment{opmerking}{\null\hfill$\diamond$}%

% http://tex.stackexchange.com/questions/22119/how-can-i-change-the-spacing-before-theorems-with-amsthm
\makeatletter
% \def\thm@space@setup{%
%   \thm@preskip=\parskip \thm@postskip=0pt
% }

\newcommand{\oefening}[1]{%
	\def\@oefening{#1}%
	\subsection*{Oefening #1}
}

\newcommand{\suboefening}[1]{%
	\subsubsection*{Oefening \@oefening.#1}
}

\newcommand{\exercise}[1]{%
	\def\@exercise{#1}%
	\subsection*{Exercise #1}
}

\newcommand{\subexercise}[1]{%
	\subsubsection*{Exercise \@exercise.#1}
}


\usepackage{xifthen}

\def\testdateparts#1{\dateparts#1\relax}
\def\dateparts#1 #2 #3 #4 #5\relax{
	\marginpar{\small\textsf{\mbox{#1 #2 #3 #5}}}
}

\def\@lesson{}%
\newcommand{\lesson}[3]{
	\ifthenelse{\isempty{#3}}{%
		\def\@lesson{Lecture #1}%
	}{%
		\def\@lesson{Lecture #1: #3}%
	}%
	\subsection*{\@lesson}
	\testdateparts{#2}
}

% \renewcommand\date[1]{\marginpar{#1}}


% fancy headers
\usepackage{fancyhdr}
\pagestyle{fancy}

\fancyhead[LE,RO]{Max Kiene}
\fancyhead[RO,LE]{\@lesson}
\fancyhead[RE,LO]{}
\fancyfoot[LE,RO]{\thepage}
\fancyfoot[C]{\leftmark}

\makeatother




% notes
\usepackage{todonotes}
\usepackage{tcolorbox}

\tcbuselibrary{breakable}
\newenvironment{verbetering}{\begin{tcolorbox}[
			arc=0mm,
			colback=white,
			colframe=green!60!black,
			title=Opmerking,
			fonttitle=\sffamily,
			breakable
		]}{\end{tcolorbox}}

\newenvironment{noot}[1]{\begin{tcolorbox}[
			arc=0mm,
			colback=white,
			colframe=white!60!black,
			title=#1,
			fonttitle=\sffamily,
			breakable
		]}{\end{tcolorbox}}




% figure support
\usepackage{import}
\usepackage{xifthen}
\pdfminorversion=7
\usepackage{pdfpages}
\usepackage{transparent}
\newcommand{\incfig}[2][0.3\textwidth]{%
	\def\svgwidth{#1}
	\import{./figures/}{#2.pdf_tex}
}

% %http://tex.stackexchange.com/questions/76273/multiple-pdfs-with-page-group-included-in-a-single-page-warning
\pdfsuppresswarningpagegroup=1

\author{Max Kiene}
